\documentclass[11pt]{article}
%Gummi|061|=)
\usepackage{amsmath}
\usepackage{amsthm}
\usepackage{amsbsy}
\usepackage{amssymb}
\usepackage{inputenc}
\usepackage{graphicx}
\usepackage{selinput}
\SelectInputMappings{%
adieresis={ä},
germandbls={ß},
}
\title{\textbf{Versuch 303: Der Lock-In-Verstärker}}
\author{Martin Bieker\\
		Julian Surmann\\
		\\
		Durchgef\"{u}hrt am 05.12.2013\\
		TU Dortmund}
\date{}
\usepackage{graphicx}
\begin{document}
\renewcommand\tablename{Tabelle}
\renewcommand\figurename{Abbildung}
\maketitle
\thispagestyle{empty}
\newpage
\clearpage
\setcounter{page}{1}


\section{Einleitung}
Im folgenden Versuch soll der Lock-In-Verstärker kennengelernt werden. Beim Lock-In-Verstärker handelt es sich um eine Schaltung, die ein stark verrauschtes Signal messbar macht.
\section{Theorie}
Um sehr schwache elektrische Signale zu messen, hat der Lock-In-Verstärker einen integrierten phasenempfindlichen Detektor. Zur Messung wird ein Signal mit einer beliebigen, aber konstanten Referenzfrequenz $w_0$ moduliert. Dieses Signal durchläuft den Messaufbau und wird dann mit einem geeigneten Empfänger wieder aufgenommen. Das aufgenommene Signal kann jetzt stark verrauscht sein, z.B. durch elektromagnetische Störquellen. Dieses Signal wird zunächst in einen Bandpass gespeist, sodass Störungen mit einer deutlich größeren $(w \gg w_0)$ bzw. kleineren Frequenz $(w \ll w_0)$ entfernt werden. Anschließend wird das Signal mit einem parallel erzeugten Referenzsignal der gleichen Frequenz im Mischer multipliziert. Die Phasenverschiebung zwischen den beiden Signalen ist dabei einstellbar, so können die Signale immer gleichphasig $(\Delta \phi) $eingestellt werden. Das bei der Multiplikation entstehende Signal wird schließlich in einen Tiefpass-Filter geführt. Dieser integriert das Signal über mehrere Perioden, sodass alle Störungen und das Rauschen fast restlos entfernt werden.
\begin{equation}
F O R M E L  
\end{equation}
\section{Aufbau und Durchf\"{u}hrung}
Hier folgt der Aufbau und die Durchführung.
\section{Auswertung}
Hier folgt die Auswertung.
Das ist eine geeignete Tabelle mit 5 Spalten:
\begin{table}[h]
\centering
\begin{tabular}{|c|c|c|c|c|}
\hline
$\varphi[^\circ]$ & $\varphi [rad]$ & $F [N]$ & $M [N]$ & $ D [\frac{Nm}{rad}]$ \\
\hline
20.0 & 0.349 & 0.1 & 0.01 & 0.0286\\
40.0 & 0.698 & 0.2 & 0.02 & 0.0286\\
\hline
\end{tabular}
\caption{Benennung der Tabelle}
\end{table}
\section{Diskussion}
Hier kommt die Diskussion hin.
\section{Literatur- und Abbildungsverzeichnis}
Hier befindet sich das Literatur- und Abbildungsverzeichnis.
\section{Anhang}
Hier stehen die im Anhang angefügten Dokumente.
\end{document}
